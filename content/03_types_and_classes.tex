%! Author = Philipp Emmenegger
%! Date = 10/06/2021

\section{Types and Classes}
\subsection{Type}
\begin{itemize}
    \item Name for a collection of related values
    \item e.g. $Bool = False | True$
    \item Every well-formed expression has a type
    \item Type can be automatically calculated at compile time: \textit{type inference}
    \item Removing the need for type checks at run time => safer /faster
\end{itemize}
\textbf{Type Error}
\begin{itemize}
    \item Applying a function to one or more arguments of the wrong type
\end{itemize}

\subsubsection{Basic Types}
\begin{itemize}
    \item \textbf{Bool}: logical values
    \item \textbf{Char}: single values
    \item \textbf{String}: strings of chars
    \item \textbf{Int}: fixed-precision int
    \item \textbf{Integer}: arbitrary-precision integers
    \item \textbf{Float}: floating-point numbers
\end{itemize}

\subsubsection{List Types}
\begin{itemize}
    \item Sequence of values of the same type
    \item Type of a list says nothing about the length
    \item Lists of lists possible
\end{itemize}
\begin{lstlisting}
[['a'], ['b','c']] :: [[Char]]
\end{lstlisting}

\subsubsection{Tuple Types}
\begin{itemize}
    \item Sequence of values of different types
    \item Type of a tuple encodes its size
    \item Type of components is unrestricted
\end{itemize}
\begin{lstlisting}
(False, 'a', True) :: (Bool, Char, Bool)
\end{lstlisting}

\subsubsection{Function Types}
\begin{itemize}
    \item Mapping from values of one type to values of another
    \item Argument and result types are unrestricted
    \item Functions with multiple arguments / results possible using lists / tuples
\end{itemize}
\begin{lstlisting}
add :: (Int, Int) -> Int
add (x,y) = x + y
\end{lstlisting}

\subsubsection{Curried Functions}
\begin{itemize}
    \item Multiple input arguments typically implemented by returning functions as results
    \item Functions that take their arguments one at a time
    \item Functions with more that two arguments can be curried by returning nested functions
\end{itemize}
\begin{lstlisting}
add' :: Int -> (Int -> Int)
add' x y = x + y
\end{lstlisting}

\subsubsection{Currying Conventions}
\begin{itemize}
    \item $->$ operator is right associative
    \item Rightmost type is the result
    \item Precending types are the input
    \item Consequence: Function application is left associative
    \item All functions in Haskell are normally defined in curried form
\end{itemize}

\subsubsection{Polymorphic Functions}
\begin{itemize}
    \item Function type contains one or more type variables
    \item Type variables must begin with lowercase letters
\end{itemize}
\begin{lstlisting}
length :: [a] -> Int
\end{lstlisting}

\subsubsection{Overloaded Functions}
\begin{itemize}
    \item Function type contains one or more class constraints
    \item Class constraints are expressed as type classes
    \item Can be instantiated to any types that satisfy the constraints
\end{itemize}
\begin{lstlisting}
(+) :: Num a => a -> a -> a
\end{lstlisting}
\textbf{Type classes in Haskell:}
\begin{itemize}
    \item Num: Numeric types
    \item Eq: Equality types
    \item Ord: Ordered types
\end{itemize}
\begin{lstlisting}
(+) :: Num a => a -> a -> a
(==) :: Eq a => a -> a -> Bool
(<) :: Ord a => a -> a -> Bool
\end{lstlisting}

\subsection{Defining Functions}
\subsubsection{Conditional Expressions}
\begin{itemize}
    \item Can be nested
    \item Must always have an else branch
\end{itemize}
\begin{lstlisting}
abs :: Int -> Int
abs n = if n >= 0 then n else -n
\end{lstlisting}

\subsubsection{Guarded Equations}
\begin{itemize}
    \item Make definitions using multiple conditions
    \item Easier to read
\end{itemize}
\begin{lstlisting}
abs n
    | n >= 0 = n
    | otherwise = -n
\end{lstlisting}

\subsubsection{Pattern Matching}
\begin{itemize}
    \item Can be defined many different ways
    \item Some ways may be more efficient
    \item $\_$: wildcard pattern, matches any argument
    \item Patterns are matched in order
    \item Patterns may not repeat variables
\end{itemize}
\begin{lstlisting}
not :: Bool -> Bool
not False = True
not True = False
\end{lstlisting}

\subsubsection{List Patterns}
\begin{itemize}
    \item Non-empty lists are constructed by repeated use of \textit{cons} operator: $(:)$
    \item Functions on lists can be defined using $x : xs$ patterns
\end{itemize}
\begin{lstlisting}
[1,2,3,4] = 1 : (2 : (3 : (4 : [])))
\end{lstlisting}


\subsection{Lambda Expressions}
\begin{itemize}
    \item Used to define anonymous functions
    \item e.g. $\backslash x -> x + x$
    \item Can give a formal meaning to functions defined using currying
\end{itemize}
\begin{lstlisting}
add x y = x + y
add = \x -> (\y -> x + y)
\end{lstlisting}
\textbf{Avoid naming functions that are only referenced once:}
\begin{lstlisting}
odds n = map f [0 .. n - 1]
            where
                f x = x * 2 + 1
odss n = map (\x -> x * 2 + 1) [0 .. n - 1]
\end{lstlisting}

\subsection{Operators}
\begin{itemize}
    \item \textbf{Prefix} notation is used for function application
    \item Is \textbf{Infix} notation desired, one may use operators (e.g. $+$)
    \item Functions can be converted into operators using backticks: $`div`$
    \item Operators can be converted to functions using brackets: $(+)$
\end{itemize}












