%! Author = Philipp Emmenegger
%! Date = 14/07/2021

\section{Higher-Order Functions}
A function is called higher-order if it takes a function as an argument or returns a function as a result.
\begin{lstlisting}
twice :: (a -> a) -> a -> a
twice f x = f (f x)
\end{lstlisting}

\subsection{Why are they useful?}
\begin{itemize}
    \item \textbf{Common programming idioms} can be encoded as functions within the language itself
    \item \textbf{Domain specific languages} can be defined as collections of higher-order functions
    \item \textbf{Algebraic properties} of higher-order functions can be used to reason about programs
\end{itemize}

\subsection{Examples}
\subsubsection{map}
Applies a function to every element of a list.
\begin{lstlisting}
map :: (a -> b) -> [a] -> [b]
-- defined using list comprehension --
map f xs = [f x | x <- xs]
-- defined using recursion --
map f [] = []
map f (x : xs) = f x : map f xs
-- for example: --
map (+1) [1,3,5,7]
[2,4,6,8]
\end{lstlisting}

\subsubsection{filter}
Selects every element from a list that satisfies a predicate.
\begin{lstlisting}
filter :: (a -> Bool) -> [a] -> [a]
-- defined using list comprehension --
filter p xs = [x | x <- xs, p x]
-- defined using recursion --
filter p [] = []
filter p (x : xs)
    | p x = x : filter p xs
    | otherwise = filter p xs
-- example --
filter even [1 .. 10]
[2,4,6,8,10]
\end{lstlisting}

\subsubsection{foldr}
\textbf{Why is foldr Useful?}
\begin{itemize}
    \item Some recursive functions on lists are easier to define
    \item Properties of functions can be proved using albegraic properties of foldr
    \item Advanced program optimisations can be simpler if foldr is used in place of explicit recurison
\end{itemize}
A number of functions on lists can be defined using the following simple pattern of recursion:
\begin{lstlisting}[escapeinside={(*}{*)}]
f [] = v
f (x : xs) = x (*$\bigoplus$*) f xs
\end{lstlisting}
Some function $\bigoplus$ is applied to the head of non-empty lists, and $f$ to its tail.
The value \textbf{v} is typically the identity element of $\bigoplus$.\\
\textbf{For example:}
\begin{lstlisting}
sum [] = 0
sum (x : xs) = x + sum xs

product [] = 1
product (x : xs) = x * product xs

and [] = True
and (x : xs) = x && and xs
\end{lstlisting}
The higher-order library function \textbf{foldr} (fold right) uses this pattern of recursion with the function $\bigoplus$ and the value \textbf{v} as arguments:
\begin{lstlisting}
-- defined using recursion --
foldr :: (a -> b -> b) -> b -> [a] -> b
foldr f v [] = v
foldr f v (x : xs) = f x (foldr f v xs)
-- example --
sum = foldr (+) 0
product = foldr (*) 1
and = foldr (&&) True
-- other examples -- 
length = foldr (\ _ n -> 1 + n) 0
reverse = foldr (\ x xs -> xs ++ [x]) [] 
-- append function (++) --
(++ ys) = foldr (:) ys
\end{lstlisting} 

\subsubsection{Other Library Functions}
\textbf{(.):} returns the composition of two functions as a single function.
\begin{lstlisting}
(.) :: (b -> c) -> (a -> b) -> (a -> c)
f . g = \ x -> f (g x)
-- example --
odd :: Int -> Bool
odd = not . even
\end{lstlisting}
\textbf{all:} decides if every element of a list satisfies a given predicate.
\begin{lstlisting}
all :: (a -> Bool) -> [a] -> Bool
all p xs = and [p x | x <- xs]
-- example --
all even [2,4,6,8,10]
True
adslnasdlasd
\end{lstlisting}
\textbf{any:} decides if at least one element of a list satisfies a predicate.
\begin{lstlisting}
any :: (a -> Bool) -> [a] -> Bool
any p xs = or [p x | x <- xs]
-- example --
any (== ' ') "abc def"
True
\end{lstlisting}
\textbf{takeWhile:} selects elements from a list while a predicate holds.
\begin{lstlisting}
takeWhile :: (a -> Bool) -> [a] -> [a]
takeWhile p [] = []
takeWhile p (x : xs)
    | p x = x : takeWhile p xs
    | otherwise = []
-- example --
takeWhile (/= ' ') "abc def"
"abc"
\end{lstlisting}
\textbf{dropWhile:} drops elements from a list while a predicate holds.
\begin{lstlisting}
dropWhile :: (a -> Bool) -> [a] -> [a]
dropWhile p [] = []
dropWhile p (x : xs)
    | p x = dropWhile p xs
    | otherwise = x : xs
-- example --
dropWhile (== ' ') " abc"
"abc"
\end{lstlisting}